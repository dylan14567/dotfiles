\documentclass[a4paper, 10pt]{article}
\usepackage[a4paper, margin=0.75in]{geometry} % Paquete para ajustar márgenes
\usepackage{xcolor}
\usepackage{nopageno} % Paquete para eliminar los números de página
\usepackage{multicol} % Paquete para las columnas múltiples
\usepackage{titlesec} % Paquete para personalizar los títulos de las secciones
\usepackage{cite}
\usepackage{graphicx} % Paquete necesario para incluir imágenes
\usepackage{float} % Paquete para controlar mejor la colocación de figuras
\usepackage[pdftex, 
pdfauthor={Autor},
pdftitle={Titulo},
pdfsubject={Lorem ipsum dolor sit amet, consectetur adipiscing elit. Duis mauris tortor},
pdfkeywords={Astronomy, Computing, Astronomy, Computing, Interdisciplinary Research, Data Analysis}]{hyperref}

% Definir un color azul más oscuro
\definecolor{tituloColor}{RGB}{70, 130, 180}

% Personalizar el tamaño de los títulos de las secciones
\titleformat{\section}
{\normalfont\large\bfseries}{\thesection}{1em}{}

% Ajustar la separación entre columnas
\setlength{\columnsep}{14.2pt} % Aproximadamente 0.5 cm

\title{\textcolor{tituloColor}{Titulo}}
\author{Nombre}
\date{} % No establece ninguna fecha

\begin{document}
	
	% Redefinir el comando \maketitle
	\makeatletter
	\renewcommand{\maketitle}{
		\begin{center}
			{\LARGE \@title \par}
			\vskip 1em
			{\large \@author \par}
			\vskip 0.5em
			{\large http://example.com \par}
			\vskip 1.5em
		\end{center}
	}
	\makeatother
	
	\maketitle
	
	% Añadir las líneas de separación en el abstract y la sección de keywords
	\vspace{0.5em} % Espacio antes del abstract
	\hrule % Línea superior
	\vspace{0.5em} % Espacio después de la línea superior
	
	% Alinear el abstract a la izquierda y ajustar el tamaño de la letra
	\noindent
	\begin{flushleft}
		\textbf{Abstract:} Lorem ipsum dolor sit amet, consectetur adipiscing elit. Duis mauris tortor, faucibus in ante id, egestas pretium orci. Ut tristique pharetra turpis, nec tristique enim porta eget. Sed eu libero varius, efficitur dui ac, aliquet turpis.
	\end{flushleft}
	
	\vspace*{-1em} % Espacio muy corto entre el abstract y los keywords
	\begin{flushleft}
		\textbf{Keywords:} Astronomy \& Computing, Astronomy, Computing, Interdisciplinary Research, Data Analysis
	\end{flushleft}
	
	\vspace{0.5em} % Espacio antes de la línea inferior
	\hrule % Línea inferior
	\vspace{1em} % Espacio antes del contenido principal
	
	\begin{multicols}{2}
		\section{Introducción}
		Lorem ipsum dolor sit amet, consectetur adipiscing elit. Donec sit amet viverra lorem. Sed elementum augue dolor, eget tempor justo porttitor at. Donec elit magna, scelerisque a tempor et, suscipit eget libero. Nunc magna lorem, tincidunt ut ligula non, mattis eleifend erat. Nullam ornare nunc ligula, ut gravida elit rutrum nec. 
                \begin{figure}[H] % Uso del entorno figure
		 	\centering
		 	\includegraphics[width=1\columnwidth]{encuesta.jpeg} % Ruta relativa a la imagen
		 	\caption{Encuesta sobre tendencias}
		 	\label{fig:encuesta}
	        \end{figure}
	 
	        Lorem ipsum dolor sit amet, consectetur adipiscing elit. Donec sit amet viverra lorem. Sed elementum augue dolor, eget tempor justo porttitor at. Donec elit magna, scelerisque a tempor et, suscipit eget libero. Nunc magna lorem, tincidunt ut ligula non, mattis eleifend erat. Nullam ornare nunc ligula, ut gravida elit rutrum nec. \cite{autor2024}
		
		\section{Material y método}
		Lorem ipsum dolor sit amet, consectetur adipiscing elit. Donec sit amet viverra lorem. Sed elementum augue dolor, eget tempor justo porttitor at. Donec elit magna, scelerisque a tempor et, suscipit eget libero. Nunc magna lorem, tincidunt ut ligula non, mattis eleifend erat. Nullam ornare nunc ligula, ut gravida elit rutrum nec.
		
		\section{Resultados}
		Lorem ipsum dolor sit amet, consectetur adipiscing elit. Donec sit amet viverra lorem. Sed elementum augue dolor, eget tempor justo porttitor at. Donec elit magna, scelerisque a tempor et, suscipit eget libero. Nunc magna lorem, tincidunt ut ligula non, mattis eleifend erat. Nullam ornare nunc ligula, ut gravida elit rutrum nec.
		
		\section{Discusión}
		Lorem ipsum dolor sit amet, consectetur adipiscing elit. Donec sit amet viverra lorem. Sed elementum augue dolor, eget tempor justo porttitor at. Donec elit magna, scelerisque a tempor et, suscipit eget libero. Nunc magna lorem, tincidunt ut ligula non, mattis eleifend erat. Nullam ornare nunc ligula, ut gravida elit rutrum nec.
		
		\section{Conclusión}
		Lorem ipsum dolor sit amet, consectetur adipiscing elit. Donec sit amet viverra lorem. Sed elementum augue dolor, eget tempor justo porttitor at. Donec elit magna, scelerisque a tempor et, suscipit eget libero. Nunc magna lorem, tincidunt ut ligula non, mattis eleifend erat. Nullam ornare nunc ligula, ut gravida elit rutrum nec.
	\end{multicols}
	
	\vspace{1em} % Espacio antes de la línea
	\hrule % Línea antes de las referencias
	\vspace{1em} % Espacio antes de las referencias
	
	\renewcommand{\refname}{Referencias} % Cambia el nombre de la sección de referencias
	\begin{thebibliography}{9}
		\bibitem{autor2024}
		Juan Pérez,
		\textit{El Estudio de Casos en Matemáticas},
		Editorial Científica, 2024.
		
		\bibitem{autor2023}
		María López,
		"Nuevas Tendencias en Física Cuántica",
		Revista de Física, vol. 42, no. 7, pp. 123-145, 2023.
		
		\bibitem{autor2022}
		Carlos García,
		"Avances en la Programación Matemática",
		Actas del Congreso Internacional de Matemáticas, pp. 567-578, 2022.
	\end{thebibliography}

        \vspace{0.5em}

        \begin{minipage}[t]{0.2\textwidth} % Alineación a la izquierda y tamaño
		\centering
		\includegraphics[width=\textwidth]{license.png} % Ruta relativa a la imagen
		\vspace{0.5em}
		\small \textcopyright 2024 Autor.
	\end{minipage}
	
\end{document}
